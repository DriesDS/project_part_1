\documentclass[a4paper]{article}
\usepackage[dutch]{babel}
\usepackage{amsmath}
\usepackage{graphicx}
\usepackage{epstopdf}
\usepackage{caption}
\usepackage{subcaption}
\usepackage{courier}

\title{Technisch Wetenschappelijke Software: Project deel 1}
\author{Dries De Samblanx}
\date{maandag 23 november 2015}

\newcommand{\opgave}[1]{\section*{Opgave #1}}
\newcommand{\dx}{\Delta x}
\newcommand{\dy}{\Delta y}
\newcommand{\dz}{\Delta z}
\newcommand{\dt}{\Delta t}

\begin{document}
\maketitle

\begin{abstract}

Het project heb ik gemaakt op een totaal van ongeveer 60 uur. In deze tijd heb ik alle opties kunnen programmeren die gevraagd waren. Een groot deel van het debug-werk stak in fouten die ontstonden door het feit dat alles in transposes ge\"implementeerd is. Door de vorm van de opgave was het gemakkelijk om van de code '\(the ground up\)' te schrijven en dan in de breedte uit te breiden. Dit was zeer nuttig aangezien het zo gemakkelijk was code snel te testen. Uit de testen is gebleken dat het werken met rank-k matrices enkel nuttig is wanneer er veel bewerkingen moeten gebeuren, wanneer de matrices op deze manier gegeven zijn of wanneer geheugen en belangrijke rol speelt. 

\end{abstract}

\section*{Methodologie}

\subsection*{ontwerp en datastructuren}

Bij een project van wetenschappelijke software is het belangrijk om niet direct te beginnen met implementeren maar om eerst na te denken over het ontwerp van de implementatie en over de gebruikte datastructuren.\\ 
Een matrix heb ik voorgesteld met behulp van een voorgedefini\"eerd type. Op deze manier is het eenvoudig om met een boolean bij te houden of een matrix volledig of met een rank-k benadering wordt bijgehouden. Daarnaast bevat het type ook twee pointers naar een twee-dimensionale array. Hiermee kan bij het inlezen van de matrix beslist worden of er \'e\'en of twee pointers gealloceerd worden en hoe groot de arrays zijn. Het gebruik van pointers boven allocatable arrays heeft ook voordelen op vlak van geheugengebruik. Bij het berekenen van een product tussen een rank-k matrix en een volledige matrix wordt het resultaat bijvoorbeeld gegeven met behulp van een rank-k matrix. \'e\'en van de twee factoren van deze uitkomst is identiek aan \'e\'en van de twee factoren van de oorspronkelijke rank-k matrix. Door het gebruik van pointers kunnen deze allebei beschreven worden door hetzelfde stuk geheugen, wat niet mogelijk was geweest met een allocatable array. \\
De structuur van de implementatie van het project lag in grote lijnen voor de hand. Het was nodig om een interface te schrijven om alle commando's door te verwijzen naar hun respectievelijke functie. Eerst had ik alle commando's vertaald in kleine programma's die zelf matrices inlazen en wegschreven. Hier ben ik achteraf op teruggekomen door het globale programma verantwoordelijk te maken voor het inlezen en wegschrijven van matrices. Dit vertaalde zich dan in \'e\'en programma dat gebruikmaakte van subroutines die bewerkingen deden op de matrices die zij megekregen hadden. Deze implementatie had niet alleen het voordeel van overzichtelijker te zijn en een indruk te geven van een beter samenhangend geheel, maar bleek ook veel handiger voor het uitvoeren van de tests waar gewerkt moest worden met vooraf ingelezen matrices. Ook zorgt dit ervoor dat de functies minder van elkaar afhangen wat over het algemeen leidt tot duidelijkere en gemakkelijker te veranderen code.

\subsection*{inlezen en wegschrijven matrices}

De functies die fortran voorziet voor het inlezen en wegschrijven van matrices zijn gemakkelijk en robuust. Omdat deze lijn per lijn inlezen en wegschrijven en omdat fortran column-major orde is, heb ik de keuze gemaakt al de matrices in hun getransponeerde vorm op te slaan. Op deze manier gebeurt het inlezen en het wegschrijven voor grote matrices sneller aangezien de elementen van de matrices op deze manier in dezelfde volgorde worden ingelezen als weggeschreven. Dit resulteert in een grote vermindering van het aantal geheugentoegangen wat een belangrijke vertragende factor is. Alle bewerkingen die op gewone matrices gedaan worden kunnen herleid worden tot bewerkingen op hun transposes. De moeilijkheid die dit met zich meebrengt is het denkwerk. Methodes voor bijvoorbeeld het product van twee rank-k matrices zijn relatief eenvoudig om over te redeneren, maar wanneer men enkel beschikt over de transposes, en ook de transposes van de uitkomstmatrix moet bekomen, moet men heel geconcentreerd blijven om geen fouten te maken. Deze beslissing heeft dan ook heel wat extra denkwerk gekost en een hele tijd extra debug werk. Daarnaast is dit deel van de code vaak te verwaarlozen qua rekenwerk. Enkel bij triviale gevallen zoals het vragen van een full matrix van een full matrix leidt dit tot merkbare performantieverbetering.\\
Voor randgevallen zoals de implementatie bij rank-0 matrices moest de compleetheid van de code afgewogen worden tegen de compactheid en de effici\"entie. Om bewerkingen met rank-0 matrices correct af te handelen zijn er echter een hele hoop if-testen nodig op verschillende plaatsen. Aangezien dit ten koste is van zowel de effici\"entie als de leesbaarheid van de code heb ik beslist dit niet te doen. Om toch ongewenste fouten te voorkomen heb ik bij de implementatie van lowrank er voor gezorgd dat er nooit een rank-0 matrix teruggegeven wordt. \\
Aangezien matrices zowel pointers als allocatable arrays kunnen bevatten is het niet vanzelfsprekend om objecten van het type Matrix te dealloceren. Daarom is er hiervoor een methode voorzien in het bestand \(matrixconverter.f90\). Dit zorgt voor minder code duplicatie wat tot eenvoudiger aanpasbare code leidt, maar verhoogt wel de samenhang tussen de verschillende bestanden. Aangezien het bestand sowieso al ge\"importeerd moest worden voor het type Matrix is er gekozen voor \'e\'en enkele functie voor het dealloceren van deze objecten.

\subsection*{matrixbewerkingen}
Alle matrices worden volledig voorgesteld in dubbele precisie. Voor wetenschappelijke software is het belangrijk bewerkingen niet uit te voeren met enkele precisie, maar zeker met dubbele. Uit de oefenzittingen is gebleken dat extended precisie bij de compiler Nagfor bijvoorbeeld verschilt van de extended precisie bij andere compilers. Daarom is heel de implementatie gebeurd met dubbele precisie. Deze beslissing is duidelijk gemaakt door het gebruik van de benaming \verb!double precision! van alle types matrices en re\"ele getallen. Dit maakt het moeilijker om een eventuele overstap te maken naar bijvoorbeeld enkele precisie getallen, maar omdat de code zo leesbaarder is en alle gebruikte subroutines van lapack gemaakt zijn voor dubbele precisie is toch deze formulering gebruikt. Ook is deze naamgeving compiler onafhankelijk.\\
Een andere belangrijke beslissing was nodig bij het berekenen van een rank-k benadering van een matrix die ingelezen wordt in rank vorm. De gebruikte implementatie gebruikt geen precondities in verband met de matrices van de rank-k benadering maar is wel niet heel effici\"ent wanneer het op geheugen of snelheid aankomt. De gebruikte methode maakt eerst de volledige matrix met de functie \verb!full! en berekent vervolgens hiervan de rank-k benadering. Wanneer op deze manier een rank-2 benadering wordt gemaakt van een rank-5 (1000x1000) matrix moet wel eerst de volledige 1000x1000 matrix berekend worden. Een andere implementatie zou bijvoorbeeld kunnen veronderstellen dat de gegeven rank matrix de vorm heeft zoals hij uit de methode lowrank zelf zou komen. Dan zou het al voldoende zijn om na te kijken welke kolommen van de matrix A bij de laagste singuliere waarden horen en deze dan samen met hun overeenkomstige kolommen in de B matrix laten vallen. Dit maakt de methode minder robuust maar het zou nog altijd een correcte implementatie zijn voor matrices gevormd door de applicatie zelf. Daarnaast zou dit veel sneller en veel geheugen effici\"enter zijn. Met deze laatste methode heeft de gebruiker nog altijd de mogelijkheid via het programma eerst zelf de volledige matrix te berekenen en hierna een nieuwe rank-k benadering te maken. Bij het berekenen van een rank-k benadering was het ook mogelijk dat de rank van de matrix intrinsiek kleiner is dan de gevraagde rank. Omdat we hier met floating point getallen werken is het hiervoor wel nodig te testen of de singuliere waarden kleiner zijn dan een bepaalde epsilon. Op nul testen zou hier immers niet werken aangezien de singuliere waarden in dit geval altijd van de grootteorde van de machine precisie zijn.

\subsection*{integraal vergelijking}

Voor het opstellen van de matrix A bij het commando makeGFull is de formule op een aangepaste versie ge\"implementeerd.
\begin{equation}
\label{matA}
\begin{aligned}
	&A_{i,j} = -ln( \\ &\sqrt[]{(cos(2\pi (j+0.5)/N)-cos(2\pi i/N))^2
	 +(sin(2\pi (j+0.5)/N)-sin(2\pi i/N))^2})
\end{aligned}
\end{equation}
We kunnen hierbij de twee kwadraten uitwerken en de log van een vierkantswortel schrijven als de halve log.
\begin{equation}
\label{matAvoluit}
\begin{aligned}
	A_{i,j} = & -0.5ln((cos(2\pi (j+0.5)/N)^2-2cos(2\pi i/N)cos(2\pi (j+0.5)/N) \\ & + cos(2\pi i/n)^2
	 +(sin(2\pi (j+0.5)/N)^2 \\ & -2sin(2\pi i/N)sin(2\pi (j+0.5)/N)+sin(2\pi i/N)^2)
\end{aligned}
\end{equation}
Nu kunnen we ook nog tweemaal gebruik maken van de grondformule van de goniometrie en van de dubbele cosinus regel.
\begin{equation}
\label{matAkort}
\begin{aligned}
	A_{i,j} = -0.5ln(2 - 2cos((1+2j-2i)\pi/N))
\end{aligned}
\end{equation}
Hieraan kunnen we ook zien dat alle kolommen van deze matrix kopie\"en zijn van elkaar, enkel \"e\"en plaats naar onder geshift per kolom. Op dezelfde manier kunnen we de sommatie voor $u_N(x)$ herschrijven.
\begin{equation}
\label{un}
\begin{aligned}
	u_N(x) = -\sum_{j=0}^{N-1} 0.5c_jln(1+x_1^2+x_2^2 &-2x_1cos(2\pi (j+0.5)/N)\\&-2x_2sin(2\pi (j+0.5)/N))
\end{aligned}
\end{equation}
Op deze manier proberen we de effici\"entie en nauwkeurigheid van het opstellen van de matrix A en de vector u te verbeteren. \\
Het oplossen van het stelsel voor SolveIntFull kan problemen geven aangezien het stelsel singulier is voor even waarden van N. Dit leidt niet direct tot foute waarden van de sommatie, maar de nauwkeurigheid is heel wat slechter. Een oplossing hiervoor zou zijn om enkel oneven waarden van N te gebruiken.

\subsection*{tests en compilatie}
Bij het uitvoeren van de tests voor het bepalen van de twee reeksen ontstaan er problemen bij het dealloceren van matrices. Door het gebruik van valgrind is het duidelijk dat er bij enkele bewerkingen geen geheugenlekken aanwezig zijn. Bij de tests worden echter dezelfde matrices gebruikt voor meerdere achtereenvolgende bewerkingen op te doen. Wanneer ik hier probeer om de matrices te deallocaren komen er fouten naar boven. Wanneer er print statements staan tussen elke regel in de functie \verb! M_dealloc! in het bestand \(matrixConverter.f90\) geeft hij weer dat de matrix gealloceerd is, maar geeft hij een error omdat hij een matrix geen twee keer kan dealloceren. Dit probleem toont zich nooit wanneer er telkens maar \'e\'en bewerking gebeurt via de command line, dus een eenvoudige oplossing om rond dit probleem te werken zou zijn om de tests zo te schrijven dat alles via de command line gebeurt. Omdat de matrices voor de tests echter niet zo groot zijn en omdat dit probleem zich niet voordoet bij bewerkingen via de command line heb ik deze implementatie niet verandert zodat er wel geheugenlekken zijn bij het uitvoeren van de tests. \\
De compilatie verloopt goed met alle compilers buiten Nagfor. Deze geeft problemen met de functie getarg. Dit kan ermee te maken hebben dat Nagfor extra bibliotheken nodig heeft voor communicatie met het systeem. De gebruikte vlaggen voor de overige compilers zijn -O3 om het compileren van de code te optimalizeren en -lblas en -llapack in de laatste stap bij het linken om de compiler te vertellen dat sommige methodes uit deze bibliotheken gebruikt zijn.

\subsection*{tijdsbesteding}

\begin{table}
\begin{center}
\begin{tabular}{|r||r|}
\hline
onderdeel & tijdsbesteding \\\hline    
\hline
softwareontwerp + makefile & 7 \\\hline
matprod + full & 12 \\\hline
lowrank & 11 \\\hline
matrices inlezen en wegschrijven & 8 \\\hline
makeGFull + solveIntFull + plotField & 8 \\\hline
tests & 8 \\\hline
verslag & 7 \\\hline
\hline
totaal & 61 \\\hline
\end{tabular}
\end{center}
\caption{tijdsbesteding weergegeven in uren}
\label{tijdsbesteding}
\end{table}

De tijdsbesteding voor dit project verdeeld onder de voornaamste onderdelen is weergegeven in tabel. Dit is ongeveer anderhalf maal de voorziene studiebelasting. Alle onderdelen zijn ongeveer apart van elkaar afgerond. Enkel de tests zijn verdeeld over alle subonderdelen.


\section*{Resultaten}

\begin{figure}
\centering
\begin{subfigure}{.48\textwidth}
	\centering
	\includegraphics[width=1\textwidth]{lowrankM.pdf}
	\caption{Rekentijd voor matrix product}
	\label{lowrank}
\end{subfigure}
\begin{subfigure}{.48\textwidth}
	\centering
	\includegraphics[width=1\textwidth]{fullprod.pdf}
	\caption{Rekentijd voor matrix product}
	\label{fullprod}
\end{subfigure}
\caption{rekencomplexiteit lowrank en matrix product met volledige $N \times N$ matrix}
\label{o3complex}
\end{figure}

zoals te zien op Figuur \ref{o3complex} is zowel het product van twee volledige matrices als het berekenen van een lowrank matrix met een gegeven volledige matrix \(O(n^3)\). Dit is zoals de verwachtingen. Als we beide figuren met elkaar vergelijken kunnen we ook opmerken dat het berekenen van een lagere rang matrix meer tijd vraagt dan het product nemen van twee volledige matrices. Dit wil zeggen dat wanneer we twee matrices willen vermenigvuldigen het weinig zinvol is om eerst een lagere rang matrix te berekenen. Het nut hiervan wordt pas bereikt wanneer we meerdere bewerkingen moeten doen met deze matrix. Daarnaast kunnen we vermelden dat de rekentijd van het berekenen van een lagere rang matrix onafhankelijk is van de rang waarin we ge\"interesseerd zijn. Dit komt omdat we telkens de volledige SVD berekenen met \(O(n^3)\) rekencomplexiteit.

\begin{figure}
\centering
	\includegraphics[width=1\textwidth]{rankfull.pdf}
	\caption{Rekentijd voor matrix product van 2 1000x1000 matrices (eventueel van rank-M benadering}
	\label{rankfullplot}
\end{figure}

Op Figuur \ref{rankfullplot} is de rekentijd afgebeeld van het matrix-matrix product met eventueel gebruik van rank-k matrices. Het product is altijd genomen van twee 1000 op 1000 matrices. De tijd wordt over het algemeen vooral bepaald door het aantal geheugentoegangen, maar bij het product van matrices is dit aantal geheugentoegangen over het algemeen evenredig met het aantal scalaire vermenigvuldigenen. Daarom zullen we dit aantal vermenigvuldigingen gebruiken om een schatting te maken van de rekencomplexiteit. Bij het product van een MxN matrix met een NxP matrix verwachten moeten MxP elementen berekend worden doormiddel van N vermenigvuldigingen. We verwachten dus complexiteit MxNxP wat zich herleid tot de \(O(n^3)\) zoals te zien in Figuur \ref{o3complex}.
\begin{equation}
\label{fullfull}
\begin{aligned}
	full*full \sim O(M\times N\times P)
\end{aligned}
\end{equation}
Wanneer we het product nemen van een MxN matrix van rank K met een NxP volledige matrix moet er enkel een vermenigvuldiging gedaan worden van een KxN en een NxP matrix. Dit herleid zich tot een complexiteit van KxNxP zoals te zien in de lineaire afhankelijkheid van het rank*full matrix product in Figuur \ref{rankfullplot}.
\begin{equation}
\label{rankfull}
\begin{aligned}
	rank-k*full \sim O(K\times N\times P)
\end{aligned}
\end{equation}
Wanneer K=1000 is rekentijd gelijk aan dat van een full*full product aangezien men in beide gevallen \(1000^3\) vermenigvuldigingen nodig heeft.\\
Wanneer we een MxN matrix van rank $K_1$ vermenigvuldigen met een NxP matrix van rank $K_2$ moeten er 2 vermenigvuldigingen gebeuren. Eerst gebeurt de vermenigvuldiging van een $K_1\times N$ matrix met een $N\times K_2$ matrix, en vervolgens het product van deze matrix met ofwel de $M\times K_1$ matrix als \(K_1 \leq K_2\) ofwel met de $K_2\times P$ matrix in het andere geval.
\begin{equation}
\label{rankrank}
\begin{aligned}
	rank-k_1*rank-k_2 \sim O(K_1\times N\times K_2) + O(K_1\times K_2\times (M of P))
\end{aligned}
\end{equation}
Wanneer we nu de k willen bepalen waarvoor de rekentijd korter is voor een product van een rank-k matrix met een rank-k matrix in het geval wanneer $K_1 = K_2 = K$ en $M=N=P$ moeten we enkel een eenvoudige gelijkheid oplossen.
\begin{equation}
\label{break-evenK}
\begin{aligned}
	M^3 &= (K\times M\times K) + (K\times K\times M) \\
	M^2 &= 2\times K^2 \\
	K &= M/\sqrt{2}
\end{aligned}
\end{equation}
Dit zien we terug in het snijpunt op Figuur \ref{rankfullplot} van de rechte voor full*full en rank*rank. Hier ligt de break-even op $K=1000/\sqrt{2} \approx 707$.

\section*{discussie}

Wanneer men berekeningen gaat maken met rank-k benaderingen in plaats van met de volledige matrices wil men dit meestal om twee redenen. Men wil de rekencomplexiteit en daarmee de uitvoeringstijd naar beneden halen en/of men wil het geheugen dat de matrices innemen beperken. Zoals te zien in Figuur \ref{o3complex} is het berekenen van deze benadering heel duur. Wanneer het geheugen geen rol speelt is dit dus enkel nuttig wanneer er veel berekeningen moeten gebeuren met een matrix of met matrices bekomen door bewerkingen met deze matrix. 

\subsection*{rekencomplexiteit}

Zoals aangetoond in de resultaten is het berekenen van een matrix-matrix product van een volledige met een rank-k benadering altijd voordeliger. Ook het resultaat van deze bewerking zal verkregen worden in de vorm van een rank-k matrix, dus wanneer hier verdere berekeningen mee gemaakt moeten worden kunnen ook deze gemaakt worden met een rank-k benadering. Dit is een voordeel maar wanneer men de rekentijden vergelijkt ziet men dat er toch al meer dan tien bewerkingen gedaan moeten worden met rank matrices om de kost te compenseren voor het maken van deze benadering.

\subsection*{geheugen}
Waar het geheugen van een volle matrix 8bytes*M*N bedraagt, is dat van een rank-k benadering van deze matrix 8bytes*(M*k+N*k). Hieruit valt gemakkelijk af te leiden dat $k<\frac{mn}{m+n}$ en in het geval van M=N $k<M/2$. Om veel voordeel uit het geheugen te halen moet $k\ll min(M,N)$. Dit kan ten nadele zijn van de nauwkeurigheid van de berekeningen, maar in het geval van grote matrices kan dit nodig zijn om de berekeningen te kunnen uitvoeren. Wanneer men echter werkt met gaussiaans verdeelde ongecorreleerde matrices is het geweten dat de verhouding van de grootste singuliere waarde tot de tweede grootste singuliere waarden heel groot wordt voor grote matrices. Wanneer men deze verhouding bekijkt van een 100x100 matrix is deze verhouding $\approx 9$, en wanneer men naar 1000x1000 matrices gaat is deze al vaak $\approx 27$ en bij 10 000x10 000 $\approx 86$. Hier kan een rank-1 matrix dus nog van redelijke nauwkeurigheid zijn terwijl het geheugenverbruik en de rekencomplexiteit enorm verminderd is.

\section*{besluit}

Ondanks de duidelijke winst voor het rekenen met rank-k matrices moet men goed nadenken of het wel nuttig is om deze benadering te maken. In veel gevallen is dit niet nodig en zou het onnodig rekenwerk vragen, maar in sommige gevallen kan dit een heel krachtig middel zijn om snel en met weinig geheugen bewerkingen uit te voeren met matrices. \\
Verbetering van de code zou er uit kunnen bestaan om de functie lowrank te optimaliseren zodat deze niet in alle gevallen eerst de volledige matrix moet berekenen om een rank-k benadering te maken van een rank-matrix. Ook zou men de implementatie van solveIntFull kunnen aanpassen zodat deze enkel gebruik maakt van oneven waarden van N.


\end{document}
